\documentclass[11p, titlepage, oneside, a4paper]{article}
% Packages
\usepackage{amsmath}
\usepackage{graphicx}
\usepackage{hyperref}
\usepackage[english,swedish]{babel}
\usepackage[
    backend=biber,
    style=authoryear-ibid,
    sorting=ynt
]{biblatex}
\usepackage[utf8]{inputenc}
\usepackage[T1]{fontenc}
%Källor
\addbibresource{mall.bib}
\graphicspath{ {./images/} }

% Ändra de rader som behöver ändras
\def\inst{Teknikprogrammet}
\def\typeofdoc{Laborationsrapport}
\def\course{Fysik 1 150p}
\def\pretitle{Laboration 2}
\def\title{Fjäderkraft och friktion}
\def\name{Oscar Tafvelin}
\def\username{oscar.tafvelin}
\def\email{\username{}@elev.ga.ntig.se}
\def\graders{Magnus Silverdal}

\begin{document}

\begin{titlepage}
		\thispagestyle{empty}
		\begin{large}
			\begin{tabular}{@{}p{\textwidth}@{}}
				\textbf{NTI gymnasiet \hfill \today} \\
				\textbf{\inst} \\
				\textbf{\typeofdoc} \\
			\end{tabular}
		\end{large}
		\vspace{10mm}
		\begin{center}
			\LARGE{\pretitle} \\
			\huge{\textbf{\course}}\\
			\vspace{10mm}
			\LARGE{\title} \\
			\vspace{15mm}
			\begin{large}
				\begin{tabular}{ll}
					\textbf{Namn} & \name \\
					\textbf{E-mail} & \texttt{\email} \\
				\end{tabular}
			\end{large}
			\vfill
            \includegraphics[width=0.5\textwidth]{images/NTI Gymnasiet_Symbol_print_svart.png}
			\vfill
            \large{\textbf{Handledare}}\\
			\mbox{\large{\graders}}
		\end{center}
	\end{titlepage}

    \begin{otherlanguage}{english}
	\begin{abstract}
        We were given the task to perform two seperate experiments. During the first experiment we needed to examine which factors would effect the friction of an object. We did this by using a real model, which consisted of a plank which we used as a inclined plane. We then took a block of wood and put it on this plank and tilted one end of the plank upwards until the block of wood started sliding with a constant speed (no acceleration). After this we measured how high up the plank had to be for this to occur which we could later use in our calculations to determine the amount of friction force and normal force the block of wood was subjected to. The experiment was repeated several times with different weights on top of the block. In the other experiment we were tasked with examining the amount of force a spring was subjected to. For this we used a real model, which consisted of a retort stand which held a spring. We then hanged a series of weights on the spring and measured the length of the spring after each weight added. After both of the experiments the measurements were written down and put in their respective formulas to calculate the results.
    \end{abstract}
    \end{otherlanguage}
    % Om arbetet är långt har det en innehållsförteckning, annars kan den utelämnas
	\pagenumbering{roman}
	\tableofcontents
	
	% och lägger in en sidbrytning
	\newpage

	\pagenumbering{arabic}
	
	% i Sverige har vi normalt inget indrag vid nytt stycke
	\setlength{\parindent}{0pt}
	% men däremot lite mellanrum
	\setlength{\parskip}{10pt}
	
	\section{Syfte och frågeställning}
		Syftet med laborationen är att analysera friktionen hos ett träblock som glider nedför ett plan och att analysera fjäderkraften hos en fjäder som hängdes från ett stativ med en laboratorieklämma. Vi vill få reda på friktionskraften och normalkraften hos träblocket och fjäderkraften hos fjädern när olika mängder vikter sätts på varje experiment.

	\section{Bakgrund och teori}
        Med utgångspunkt från experimentet kan man använda mätvärdena från experiment 1 tillsammans med .... och mätvärdena från experiment 2 kan användas tillsammans med formeln: $F_m = \times{m}{g}$ för att räkna ut fjäderkraften, (samma sak som gravitationskraften). $a_m = \frac{\Delta v}{\Delta t}$  $\frac{1}{25}$
	

	\section{Metod och materiel 1}
        \begin{enumerate}
            \item Stativ
            \item Planka (Lutande plan)
            \item Linjal
            \item Träblock
            \item Vikter
        \end{enumerate}

Det lutande planet monteras på ställningen så att den ena änden är 1 dm över bordet, se figur \ref{fig:lutandeplan}. Linjalen placeras längs planet så att det finns en längdskala  i filmen. Kameran placeras vid sidan av uppställningen på ett avstånd så att hela rörelsen ryms i filmen utan att kameran behöver flyttas. Vagnen rullas nedför planet samtidigt som kameran filmar rörelsen. Försöket placeras så att ljusförhållanden och bakgrund ger en så tydlig och skarp film som möjligt.

	\section{Metod och materiel 2}
        \begin{enumerate}
            \item Stativ
            \item Fjäder
            \item Linjal
            \item Vikter
        \end{enumerate}

        

        
        \begin{figure}[!h]
            \includegraphics[width=0.8\textwidth]{images/lutandePlan.jpg}
            \caption{En blid hade varit superbra här}
            \label{fig:lutandeplan}
        \end{figure}
        
        Filmen analyserades sedan med mjukvaran Tracker för att få fram en tabell med positionen som funktion av tiden.
    \newpage
	\section{Analys och beräkning}
        Datat från analysen av filmen visas i tabell \ref{table:result}
    
        
        \begin{table}
            \begin{center}
            \begin{tabular}{ |c|c| } 
                \hline
                Position (m) & Tid (s)  \\ 
                \hline
                0 & 0  \\ 
                0.1 & 0.02 \\
                \vdots & \vdots \\
                \hline
            \end{tabular}
                \caption{Mätvärden}
                \label{table:result}
            \end{center}
        \end{table}            
        

    Datat importeras i Excel och hastigheten beräknas med hjälp av formeln
    \begin{equation}
        v_m = \frac{\Delta s}{\Delta t}
    \end{equation}
    
    \section{Slutsats och resultat} 
        Resultatet av beräkningarna illustreras i graferna 2 och 3
    \section{Diskussion} 
    Resultatet är perfekt...

    
    \printbibliography

\end{document}

